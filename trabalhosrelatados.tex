\section{Trabalhos relatados}

No artigo \textit{Before We Knew It An Empirical Study of Zero-Day Attacks In The Real World} \cite{Bilge:2012}, foi desenvolvido uma 
técnica para identificar e analisar ataques de \textit{zero-day} a partir dos dados disponíveis do \textit{World Intelligence Network 
Environment} (WINE), em que identificava arquivos executáveis que estão ligados as exporações de vulnerabilidades conhecidas. Os resultados obtidos foram que 60\% das vulnerabilidades que identificou ainda não eram conhecidas. 

David Last em seu artigo \textit{Forecasting Zero-Day Vulnerabilities} \cite{Last:2016}, propõe gerar previsões de descoberta de 
vulnerabilidades nos níveis global e categoria (navegador, sistema operacional e player de vídeo), a partir de quatro conjuntos de dados 
sendo elas GlobalVuns, BrowserVulns, OSVulns e VideoVulns. O mesmo define três modelos de suítes, em que a primeira 
seria a combinação de três modelos de previsão que seria o \textit{Composite Regression}, modelo de regressão linear e modelo de 
regressão combinada. O segundo modelo de suíte seria a aprendizagem de máquinas sobre vulnerabilidades cumulativas, em que utilizou-se o 
Weka para analisar dados entre janeiro de 2000 até o início do período de previsão. Por fim, o último foi a aprendizagem de máquinas em 
vulnerabilidades mensais, em que é similiar a anterior, exceto que treina e prevê dados mensais de vulnerabilidades de software em vez de 
dados cumulativos. Os resultados no \textit{Composite Regression}, apresentaram desmepenhos melhores, e conjuntos de vulnerabilidades 
acumuláveis e mensais há um sobreposição de dados/sem sobreposição de dados. Já com as duas últimas suítes Last obteu melhores resultados 
para do OSVulns em modelos cumulativos de Aprendizagem de Máquinas, e BrowserVulns e VideoVulns, os melhores resultados foram obtidos nos 
modelos mensais Aprendizagem de Máquina.

No artigo \cite{Kumar:2016}, utilizou-se técnicas baseadas em estatísticas, assinaturas, comporamento e híbrida. O objetivo delas seriam 
identificar as façanhas em tempo real e tentar eliminar em tempo real, separação de ataque específico ou minimizar os danos provocados 
pelo ataque. Na técnica baseada em estática qualquer coisa que está acontecendo de fora do normal é bloqueada ou marcada. A segunda 
técnica é a baseada em assinaturas, porém requer uma assinatura para estar na biblioteca de assinaturas para a detecção. Dentro dessa 
técnica, temos outras três subtécnicas que seriam assinatura baseado em conteúdo, que compara o conteúdo dos pacotes com assinaturas
mal-intencionadas conhecidas, assinatura baseada na semântica são caras para gerar e não podem ser implementadas, e não podem ser 
implementadas em sistema de detecção de intrusão e a última subtécnica é assinatura baseada em vulnerabilidades, em que testamos o ponto 
de vulnerabilidades, indicando uma mensagem de entrada fará com que a execução do programa atinja o ponto de vulnerabilidade. A penúltima 
técnica é prever o comportamento futuro de um servidor web, máquina vítima, a fim de negar quaisquer comporamentos que não são esperados. 
Por fim, a última técnica seria a combinação das anteriores derrubando a fraqueza em qualquer filtro de tráfego suspeito.

Em \textit{0-Day Vulnerabilities and Cybercrime} \cite{Fotiet:2015}, é realizado várias entrevistas que mostram sua ideias, relatam sobre 
os mercados de vulnerabilidades existentes, fábricas de \textit{malwares}, cibercrime e o cibercrime em cenário militares. No penúltimo 
artigo \cite{Egelman:2013} assim como no anterior relatou mais sobre os mercados de \textit{zero-day}. Enfim, o último \cite{Luo:2015} é 
apresentado uma ferramenta, em que detecta segmentos de código que são os mesmo com patches no banco de dados, identifica \textit{patches}
que devem ser corrigidos e ajustá-los. Com essa aplicação gerou-se rapidamente um \textit{patch} de código vulnerável de um dia, e um 
exemplo citado é que foi gerado 1187 pachtes que provavelmente haviam vulnerabilidades reais, analisando mais de 2 bilhões de linhas de 
código.
