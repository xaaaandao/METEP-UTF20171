\section{Trabalhos relatados}
O tema ataque de dia zero tem sido pesquisado, atualmente, sob diferentes perspectivas. \cite{Last:2016} realizou um trabalho de busca de modelos de descoberta de vulnerabilidades que possam fornecer previsões para a descoberta de vulnerabilidades de dia zero. A partir de dados do banco de vulnerabilidade \textit{National Vulnerability Database}, o pesquisador propôs três processos distintos de previsão de vulnerabilidades de dia zero, um baseado em regressão e dois em aprendizagem de máquinas, e os resultados iniciais do estudo foram promissores.

Considerando a grande quantidade de aplicações \textit{Web} lançadas nos últimos tempos, e o aumento de ataques de dia zero contra vulnerabilidades nesses aplicativos, \cite{Kumar:2016} analisaram técnicas que permitam identificar e eliminar \textit{exploits} em tempo real ou minimizar os danos causados por eles. Na técnica baseada em estatística, qualquer acontecimento fora do normal é bloqueado ou marcado. Na técnica baseada em assinatura, usada em antivírus, a detecção depende da existência de uma assinatura previamente conhecida, e por isso não é eficaz para prevenir ataques de dia zero. A técnica baseada no comportamento objetiva prever o futuro comportamento de um servidor \textit{Web}, para negar quaisquer comportamentos que não sejam esperados. A técnica híbrida combina as técnicas baseadas na estatística e no comportamento, e visa detectar \textit{worms} polimórficos de dia zero.

\cite{Bilge:2012} propuseram uma técnica para identificar, automaticamente, ataques de dia zero, a partir de dados disponíveis no \textit{World Intelligence Network Environment}. Eles registraram quantos binários benignos e maliciosos foram baixados em 11 milhões de hosts reais em todo o mundo, e identificaram 18 vulnerabilidades exploradas antes da divulgação, sendo que 11 tinham sido empregadas em ataques de dia zero.

\cite{Fotiet:2015} analisaram as vulnerabilidades de dia zero no contexto mais amplo do cibercrime e dos mercados econômicos. A partir de entrevistas de especialistas, descreveram diferentes mercados de vulnerabilidades de zero dia, ou seja, mercados brancos, negros e cinzas (governamentais), e discutiram características das fábricas de malware e seus principais clientes. Outro trabalho que discutiu a questão do comércio dos \textit{exploits} de dia zero foi realizado por \cite{Egelman:2013}. Estes pesquisadores alertaram para a necessidade de se discutir questões éticas relacionadas aos mercados relacionados aos ataques de dia zero, visto que mercados que facilitam a venda de detalhes de vulnerabilidades estão ganhando popularidade no mundo comercial.
