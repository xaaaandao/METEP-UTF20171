\section{Referencial teórico}
O ataque de \textit{zero-day} é um ataque cibernético que explora uma vulnerabilidade que não foi divulgada publicamente, e com isso não 
há defesa contra esse ataque, consequentemente o \textit{software} não pode ser corrigido e os antivírus não detectam devido a não 
ter assinatura \cite{Bilge:2012}. O mesmo autor citado anteriormente, relata que identificou-se 18 vulnerabilidades no mundo real antes 
da divulgação em que dessas, 11 não eram anteriormente conhecidos por terem sido empregados em ataques \textit{zero-day}, sugerindo que 
esse tipo de ataque é mais comum do que se pensava. 

Os ataques de \textit{zero-day} contra aplicações Web aumentaram ~\cite{Kumar:2016}. Alguns argumentos utilizados, são que fornecedores 
pagam US\$ 60,000 dólares de recompensa na descoberta de novas vulnerabilidades em seus produtos \cite{Egelman:2013}, mais de US\$ 
100,000 dólares dependendo da complexidade da vulnerabilidade\cite{Bilge:2012}, falhas vendidas para a \textit{National Security Agency} 
(NSA) por US\$ 50,000 dólares, e com esses valores crescente justificam sendo uma motivação primária para a crescente popularidades do 
mercados de \textit{zero-day} \cite{Egelman:2013}. Além do que, essas práticas de programas de recompensas de \textit{bugs} existiam já 
em 1995 \cite{Egelman:2013}. Um outro motivo citado são quando os \textit{hackers} utilizam o \textit{software} ou por \textit{honeypots} 
\cite{Fotiet:2015}. Esses \textit{honeypots} seriam uma maneira de notar as ações que podem ser analisadas, observadas e compreendidas
\cite{Spitzner:2003}. Também podemos ter que as vulnerabilidades podem ser resultados da configuração incorreta de sistemas 
computacionais cada vez mais complexos, descuido dos usuários do sistema, como o uso de senhas fracas ou o compartilhamento de senhas com 
colegas e amigos \cite{Fotiet:2015}.

Inúmeras soluções são citadas como: gerar previsões de descoberta de vulnerabilidade nos níveis global e categoria 
(navegador, sistema operacional e \textit{player} de vídeo) \cite{Last:2016}, Teste de Penetração de Aplicações na Web (WAPT), 
\textit{Host Intrusion Protected Software} (HIPS), antivirus, atualizações de aplicações, navegadores e bloquear anexos de arquivo
de e-mail \cite{Kumar:2016}, firewall \cite{Last:2016}, identificar técnicas de arquivos executáveis que estão ligados a explorações de 
vulnerabilidades conhecidas \cite{Bilge:2012} e através de ferramentas como: o PatchGen em que detecta segmentos de códigos que são os 
mesmos com os \textit{patches} do banco de dados e em seguida identifica os \textit{patches} que devem ser corrigidos e ajustá-los 
\cite{Luo:2015}.
