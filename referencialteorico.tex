\section{Referencial teórico}
Os ataques de segurança são uma preocupação que têm aumentado muito, na última década. Em 2003, havia cerca de 6,3 bilhões de pessoas vivendo no planeta e 500 milhões de dispositivos conectados à Internet. Sete anos depois, em 2010, o número de dispositivos conectados à Internet aumentou para 12,5 bilhões, em decorrência do uso dos \textit{smartphones} e \textit{tablets}. Em 2020, estima-se que haverá 50 bilhões de dispositivos conectados à Internet \cite{Evans:2011}. Esses dados dão uma ideia de como os riscos aos ataques de segurança são altos, atualmente, e tenderão a crescer, nos próximos anos. Por exemplo, somente no primeiro trimestre de 2012, foram registrados 7.000 exemplares de ameaças (\textit{malware}) móveis ao Android \cite{McAffe:2012}.

Os ataques costumam ocorrer na Internet com variados objetivos e alvos. Todo serviço, computador ou rede acessada via Internet pode se tornar alvo de um ataque, assim como qualquer computador que tenha acesso à Internet pode ser usado em um ataque. Diferentes técnicas podem ser utilizadas nos ataques, como, por exemplo, interceptação de tráfego e negação de serviço \cite{CERT:2012}.


A “negação de serviço” (\textit{Denial of Service} – DoS) consiste em uma técnica em que o atacante utiliza um computador para tirar de operação um serviço, um computador ou uma rede conectada à Internet. Isso é realizado não para invadir ou coletar informações, mas para que usuários não possam acessá-los, por estarem indisponíveis ou extremamente lentos. Na “interceptação de tráfego” (\textit{sniffing}), dados que trafegam em redes de computadores são inspecionados, por meio de programas denominados \textit{sniffers}, com o objetivo de capturar senhas, números de cartões de crédito ou conteúdos de arquivos confidenciais \cite{CERT:2012}.

Além desses ataques tradicionais, existem técnicas de ataques direcionados, criados para invadir uma determinada rede, ou roubar informações específicas, sem serem detectados. Um exemplo é o \textit{exploit}, que significa, literalmente, “explorar”. Um \textit{exploit} é uma técnica que permite tirar proveito de uma falha existente em um \textit{software}, provocando comportamentos não esperados, para conseguir vantagens, controle do sistema ou negar serviços (DoS). Um \textit{exploit} pode ser, por exemplo, um programa executável, uma mensagem em um determinado protocolo de rede ou uma mensagem escondida em um e-mail. Geralmente, um \textit{exploit} pode ser utilizado para tomar vantagem de uma única vulnerabilidade do \textit{software}. Quando um \textit{exploit} é divulgado, a vulnerabilidade é corrigida por meio de uma correção (\textit{patch}), o que faz com que ele se torne obsoleto para as novas versões desse \textit{software}. É por isso que muitos \textit{hackers} não publicam os seus \textit{exploits}, mantendo-os privados. Esses \textit{exploits} não divulgados publicamente são chamados de dia zero \cite{Pedrosa:2007}.

Um ataque de \textit{exploit} de dia zero (\textit{zero-day exploit}) explora uma vulnerabilidade de um software ainda desconhecida por seus desenvolvedores, e por isso correções ainda não estão disponíveis \cite{Egelman:2013}. Dessa forma, o \textit{software} não pode ser corrigido e os antivírus não podem detectá-lo. Um ataque de dia zero dura, em média, 312 dias. Entretanto, existem ataques que permanecem desconhecidos por até dois anos e seis meses \cite{Bilge:2012}. 

Grande parte dos sistemas de detecção de intrusões é baseado em assinaturas, e por isso não consegue identificar ataques de dia zero. Isso porque esses sistemas somente conseguem detectar ataques para os quais já têm assinaturas, ou seja, que estejam programados para reconhecer. Dessa forma, uma rede pode estar sendo atacada, mas talvez seja necessário alguns dias até que uma nova assinatura, que detecte o ataque, seja publicada. O problema é que as empresas não podem ficar esperando pelo desenvolvimento e instalação de assinaturas, pois as ameaças se propagam rapidamente. Por exemplo, em 2003, em apenas 26 dias o \textit{worm} Blaster atacou os micros que utilizavam os sistemas operacionais Windows, que não haviam realizado a correção publicada pela Microsoft. Portanto, manter o sistema sempre atualizado é importante para prevenir ataques \cite{McCarthy:2004}.

A combinação de sistemas de detecção de intrusões baseados em assinaturas e em protocolos é a ideal para prevenir ataques. A detecção de anomalias de protocolo é feita no protocolo do aplicativo, concentrando-se na estrutura e conteúdo das comunicações. Os protocolos Telnet, HTTP, RPC e SMTP são alvos de muitos ataques. Erros de programação, como, por exemplo, o estouro do buffer, pode ser utilizado pelos invasores para comprometer ou danificar o sistema \cite{McCarthy:2004}.

Os sistemas de detecção de intrusões, sejam de assinaturas ou de protocolo, não são capazes de identificar todos os ataques. Se as empresas tiverem implementado apenas assinaturas IDS por toda a rede, não conseguirão identificar os ataques de dia zero. Considerando que as anomalias de protocolo conseguem detectar ataques desconhecidos, é preciso que as empresas reforcem suas defesas nos gates de sua rede, como, por exemplo, na conexão com a Internet, nas conexões VPN e nas conexões de rede dos clientes. Dessa forma, será possível levantar a primeira linha de defesa nos pontos de entrada, de forma a contribuir para que os ataques sejam detectados o mais cedo possível \cite{McCarthy:2004}.

Considerando o número crescente das vulnerabilidades nos softwares, organizações têm elaborado bases que integram e fornecem informações sobre essas falhas. Essas bases contribuem para que desenvolvedores construam \textit{softwares} mais seguros e mantenham os antivírus atualizados \cite{Santos:2017}. Exemplos de base de dados são a \textit{National Vulnerability Database} (NVD), do órgão do governo norte-americano \textit{National Institute of Standards and Technology}, e a \textit{Common Vulnerabilities and Exposures} (CVE), da organização norte-americana sem fins lucrativos MITRE, e a \textit{A-Z Listing of Threats \& Risks}, da empresa americana Symantec. A CVE fornece atualizações, em tempo real, sobre ameaças atuais e potencialmente futuras, níveis de gravidade e sugestões de correções a serem utilizadas em atualizações \cite{Bertucci:2017}. A NVD oferece informações de quase todas as vulnerabilidades de softwares divulgadas publicamente, e o registro de informações da severidade e facilidade de explorar essas vulnerabilidades \cite{Last:2016}.
