\section{Método}
Para a prevenção de ataques dia zero será realizada a análise da combinação de dados de vulnerabilidades já divulgadas, em três bases: o National Vulnerability Database – NVD \cite{NIST:2017}, a Common Vulnerabilities and Exposures – CVE \cite{MITRE:2017} e a A-Z Listing of Threats \& Risks \cite{Symantec:2017}.

A análise dessas três bases de dados sobre vulnerabilidades evidencia algumas peculiaridades das informações fornecidas. A \textit{A-Z Listing of Threats \& Risk} apresenta mais detalhes que as demais, sobre os tipos de vulnerabilidades: \textit{worm}, vírus, \textit{spyware}, macro, \textit{trojan} ou \textit{trojan horse} etc. Além disso, é possível: visualizar o sistema afetado; a data de descoberta; o que essa vulnerabilidade realiza, com muitos detalhes; se arquivos são criados por causa dessa falha; o risco de impacto; e algumas descrições de como determinadas vulnerabilidades podem ser reproduzidas.

A base Common Vulnerabilities Exposures (CVE) fornece uma lista de vulnerabilidades mais completa que a National Vulnerability Database (NVD). Os registros da NVD contêm informações sobre: a gravidade e facilidade de exploração de vulnerabilidades divulgadas; a complexidade de acesso, se há ou não necessidade de autenticação, os tipos de impacto e as vulnerabilidades presentes em softwares, por fornecedores. A base CVE fornece uma breve descrição da vulnerabilidade e algumas referências de como reproduzi-las, ou os arquivos que são criados a partir da vulnerabilidade. Como na base CVE são fornecidas poucas informações sobre algumas vulnerabilidades, é preciso fazer uma pesquisa a partir de seu identificador, para coleta de mais dados. Em alguns casos, a CVE não fornece descrições da vulnerabilidade, visto que o fornecedor ainda não as forneceu, e por isso é importante que os dados dessa base sejam sempre atualizados.

As informações dessas três bases são complementares, e a combinação das mesmas pode ser útil para a prevenção de ataques cibernéticos, como os de dia zero. Por exemplo, reproduzindo os ataques descritos nas bases, podemos verificar quais aplicações apresentam essas falhas, quais fornecedores têm softwares mais vulneráveis. Isso permitirá que o usuário opte por um sistema que não possua tantas vulnerabilidades. 

Após a realização de um ataque, pode-se verificar se algum arquivo foi criado, que realize sequestro de dados. Também é possível averiguar se ataques de dia zero, já divulgados e que causaram prejuízos em algumas aplicações, causam danos no software testado. Caso o resultado seja positivo, será possível avisar o fornecedor da vulnerabilidade, para que atualizações sejam feitas.