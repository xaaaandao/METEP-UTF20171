\section{Método}
A prevenção de ataques do tipo \textit{zero-day}, irá ocorrer pela análise da combinação de dados 
de vulnerabilidades que já foram divulgadas. Iremos utilizar três sites que divulgam essas 
informações, que são: o \textit{National Vulnerability Database} (NVD), \textit{National Institute 
of Standards and Technology} (NIST), \textit{Common Vulnerabilities and Exposures} (CVE) e
\textit{Listing of Threats \& Risk}, da Symantec.

Analisando essas três fontes de informações percebemos algumas pecularidades das informações 
obtidas. A \textit{List of Threats \& Risk} é a que possui mais detalhes que as demais em que 
temos o tipo vulnerabilidade, podendo ela ser um \textit{worm}, vírus, \textit{spyware}, macro, 
\textit{trojan} ou \textit{trojan horse}, etc. Além do que, nessa mesma lista conseguesse 
visualizar o sistema que é afetado, a data de descoberta, o que essa vulnerabilidade realiza com
muitos detalhes, se arquivos são criados devido a essa falha, o risco de impacto e alguns casos como 
que se reproduz essa vulnerabilidade. 

Com relação as informações do \textit{National Vulnerability Database} (NVD), são identificados por meio de
\textit{Common Vulnerabilities Exposures} (CVE) que é um dicionário de nomes comuns para vulnerabilidades de segurança 
cibernética conhecidas publicamente \cite{MITRE:2017}. As informações contidas tem informações relevantes e algumas que 
são similares da bases de dados anterior, os campos são complexidade de acesso, autenticação, tipo do impacto e também 
conseguimos também visualizar por fornecedores de \textit{software} quais falhas que ocorreram em suas aplicações. 
Porém, não existem todos os identificadores e os fornecdores no \textit{National Vulnerability Database}, somente alguns estão presente.
Devido a esse fato fomos atrás dos demais identificadores, que está presente no site \textit{Common Vulnerabilities Exposures}, 
entretanto um problema que achamos é que cada identificador possuem uma breve descrição e algumas referências, sendo que algumas delas 
possuem links que mostram os erros causados.

Acreditamos que combinar essas informações desses três conjuntos complementam uns aos outros, e julgamos que 
o resultado disso pode ser uma maneira muito útil para prevenir ataques cibernéticos como o ataque 
\textit{zero-day}. Por exemplo, reproduzindo esses ataques contidos nas informações extraídas 
podemos verificar se as aplicações tratam das vulnerabilidades, saber quais sistemas ou fornecedores 
de \textit{software} possuem mais falhas do que outros, assim fazendo com o usuário opte por uma outra alternativa que 
não possua tantas vulnerabilidades, já que se escolher ele será mais um alvo para os criminosos. Além disso, podemos complementar a 
reprodução de um ataque veirificando se alguma arquivo foi criado, depois de realizado o ataque, já que podemos ter arquivos que 
executem determinadas ações dentro de uma aplicação. Enfim, também podemos averiguar se os ataques de \textit{zero-day} que já foram 
divulgados e que causaram algumas prejuízos em algumas aplicações, causam danos no \textit{software}, e caso seja positivo o fornecedor 
poder lançar atualizações para tentar solucionar a vulnerabilidade.

Como dito anteriormente que no \textit{National Vulnerability Database} (NVD), temos uma quantidade menor de 
vulnerabilidades com relação ao \textit{Common Vulnerabilities Exposures} (CVE), podemos utilizar os 
identificadores para pesquisar mais detalhes das vulnerabilidades, de como reproduzir, quais são os prejuízos 
causados, etc. Assim como ter as aplicações no nosso sistema atualizado na versão mais recente, garante uma certo nível de proteção ter 
esses dados sempre mais atualizados fará com que as aplicações fiquem menos vulneráveis, já que existem inúmeros identficadores que 
irão receber novas vulnerabilidades que irão ser divulgadas.