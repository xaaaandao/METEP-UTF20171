\section{Resultados esperados}
Os resultados que esperamos é que conseguimos prevenir os ataques \textit{zero-day}, mas estamos cientes de
que alguns ataques não iremos conseguir prevenir, já que só iremos saber depois qual do ataque qual vulnerabilidade foi 
usada para efetuar o crime. Além do que, esperamos que essa seja mais uma maneira de prevenção, e vale a pena ressaltar 
que ainda é válido utilizar \textit{firewall}, sensores, teste de penetração, antivírus, atualizações de navegadores e 
\textit{software} para prevenir esse ataque e os demais ataque.

Além de precaver desejamos auxiliar os fornecedores de \textit{software} a não lançarem aplicações com 
vulnerabilidades e sim testá-las antes e caso necessite de correção, que seja corrigido antes de disponibilizar ela, e que fornecedores que tem aplicações já sendo utilizadas que testem elas e que caso haja alguma anomalia que 
lançem atualizações para suas aplicações evitando que seus dados ou os dados dos usuários que utilizam sejam 
expostos a pessoas não-autorizadas. 
