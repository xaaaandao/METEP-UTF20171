% 10 lines
\begin{resumo}
Os crimes cibernéticos são crimes cometidos através do uso da Internet ou de outras redes de computadores. Um tipo de crime que está 
sendo cometido é o ataque \textit{zero-day}, em que se explora vulnerabilidades que não foram divulgadas. Devido ao fato dessas falhas 
não serem expostas, queremos auxiliar usuários que tem aplicações, podendo elas serem \textit{Web} ou não, prevenir que as suas 
aplicações não fiquem expostas, e consequentemente que seus dados sigilosos não sejam divulgados as pessoas não-autorizadas. Para tal 
façanha, iremos utilizar os detalhes das falhas que foram divulgadas e a partir das informações presentes iremos utilizar as mais 
relevantes, ou tentar combinar essas informações. Com as associações feitas, desejamos testar a aplicação para verificar se existe 
alguma vulnerabilidade a partir de falhas existentes, e indicar ao fornecedor de \textit{software} quais ataques daqueles que foram 
divulgados podem ser explorados em sua aplicação. Com essa prática de prevenção desejamos que os fornecedores de \textit{software} tenham 
mais uma forma de testarem sua aplicações atuais e novas prevenindo ele de possíveis ataque de \textit{zero-day}.
\end{resumo}
