%% 10 lines
\begin{abstract}
\nohyphens{
asddasdasdsa
%Context
%Objetivo
%Resultados
}
\end{abstract}


% 10 lines
\begin{resumo}
Os crimes cibernéticos são crimes cometidos através do uso da Internet ou de outras redes de computadores. Um tipo de crime que está 
sendo cometido é o ataque \textit{zero-day}, em que se explora vulnerabilidades que não foram divulgadas. Devido ao fato dessas falhas 
não serem expostas, queremos auxiliar usuários que tem aplicações, podendo elas serem \textit{Web} ou não, prevenindo que as suas 
aplicações não fiquem expostas, e consequentemente que seus dados sigilosos não sejam divulgados as pessoas não-autorizadas. Para tal 
façanha, iremos utilizar os detalhes das falhas que foram divulgados e a partir das informações presentes iremos utilizar as mais 
relevantes, ou tentar combinar informações. Com as associações feitas, conferir as aplicações que tiveram vulnerabilidades 
expostas, e indicar ao fornecedor de \textit{software} os resultados para que a sua aplicação não sofra de ataques de \textit{zero-day}, 
ou seja, que não sejam vítimas de ataques que não foram divulgados. Com essa prática de prevenção desejamos que uma parte dos crimes cibernéticos diminuam, e que aplicações atuais e novas não sofram do ataque de \textit{zero-day}.
\end{resumo}
