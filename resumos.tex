% 10 lines
\begin{resumo}
Os crimes cibernéticos são delitos realizados por meio de computadores ou da Internet. Um tipo de crime que está aumentando nos últimos tempos é o ataque de dia zero, que explora vulnerabilidades ainda não divulgadas publicamente. Considerando que esses ataques sempre causam grandes prejuízos, o trabalho aqui proposto tem como objetivo auxiliar usuários a prevenir que suas aplicações fiquem expostas e, consequentemente, que seus dados sigilosos sejam divulgados. Para isso, serão utilizadas informações de falhas divulgadas em bases de dados sobre vulnerabilidades. Serão selecionadas as informações mais relevantes, que poderão ser combinadas. Feitas essas associações, pretende-se testar a aplicação para verificar a existência de vulnerabilidades. 
\end{resumo}
