\section{Introdução}
O crime cibernético é uma forma de crime que inclui crimes que são cometidos através do uso da Internet ou de outras redes de
informáticas \cite{Fotiet:2015}. O aumento da disseminação de crimes por computador deve se a \textit{Return On Investement}
(ROI) derivado desse tipo de atividade criminosa \cite{Fotiet:2015}. Existem vulnerabilidades que podem valer mais de 100.000 dólares, 
dependendo da sua complexidade \cite{Bilge:2012}, um exemplo que podemos citar é as vulnerabilidades do tipo \textit{zero-day}, que 
iremos explicar o que elas são logo abaixo, em que devido a esse retorno de investimento os valores variam, na qual no segundo dia ela 
pode valer um quarto do que valia no \textit{zero-day} e no décimo dia o valor dela vale 1/1000 \cite{Fotiet:2015}. Além do que, essas 
vulnerabilidades são muito desejadas por agências governamentais, cibercriminosos e empresas de software \cite{Kumar:2016}.

Um dos tipos de ataque existente é o ataque de \textit{zero-day}, em que acredita-se que o período em que provavelmente ocorreram as 
primeiras vulnerabilidades de \textit{zero-day} aconteceram nos anos 80 ~\cite{Fotiet:2015}. Esse ataque cibernético que explora uma 
vulnerabilidade que não foi divulgada publicamente, e esse ataques visam um número de organizações que possuem informações confidenciais 
que podem ser roubadas ~\cite{Bilge:2012}. Essas vulnerabilidades permanecem desconhecidas até por 2.5 anos ~\cite{Bilge:2012}. Enquanto 
a vulnerabilidade permanece desconhecida, o software obtido não pode ser corrigido e os produtos antivírus não conseguem detectar o 
ataque através da verificação baseada em assinatura ~\cite{Bilge:2012}. 

Uma maneira de prevenção é a implementação de sensores ou defesas cibernéticas, \textit{firewall} ~\cite{Last:2016}. Outras maneiras de 
prevenção seria por meio de \textit{Host Intrusion Protected Software} (HIPS), em que não se depende de assinaturas, já que há um 
monitoramente das atividades da máquina, usar um bom antivírus protegendo de ataques conhecidos e desconhecidos e atualizar suas 
aplicações sempre que houver notificação de atualização de \textit{software} ~\cite{Kumar:2016}. Usuários que utilizam o antivírus podem 
ter mais cuidados com a segurança de seus computadores, e podem ser menos expostos aos ataques ~\cite{Bilge:2012}. Existem testes de 
penetração como o \textit{Web Application Penetration Testing} (WAPT)~\cite{Kumar:2016}. Apesar de todas essas maneiras de prevenir é 
díficil, porque a informação só estará disponível quando o ataque tiver sido concluído ~\cite{Kumar:2016}. Além disso, os fornecedores 
de \textit{software} dão preferência de corrigir vulnerabilidades que foram divulgadas ou estão prestes a declarar ~\cite{Bilge:2012}.

Devido ao fato de que nesse tipo de ataque às falhas não são expostas, queremos prevenir à partir de vulnerabilidades que já foram 
divulgadas para informar aos fornecedores que possuem ou que irão lançar um possível \textit{software}, quais falhas estão ocorrendo ou 
já ocorreram, para que os mesmos possam prevenir e não sejam vítimas do ataque \textit{zero-day}. Essa prevenção ocorreria por de meio de 
uma análise de dados, em que iriamos combinar os dados que possuem mais relevância, e em que essas informações são obtidas por meio de 
sites que divulgam esses detalhes, como o \textit{National Vulnerability Database} (NVD), do \textit{National Institute of Standards and 
Technology} (NIST) e o \textit{Common Vulnerabilities and Exposures} (CVE).

Como foi dito anteriormente os crimes cibernéticos vem aumentando devido ao retorno financeiro. Acreditamos que o fato de que a cada dia 
termos uma nova aplicação, faz com que os crimes continuem crescendo, e com a prevenção do ataque \textit{zero-day} e de qualquer outro 
tipo de ataque pode evitar com que as informações presente neles, sejam expostas as pessoas que não deveriam ter acesso. Além do que, 
julgamos que precaver pode impedir que tenhasse prejuízos financeiros, já que nesses dados podem conter informações de senhas de bancos. 
Aliás, ter essa cautela seria uma forma dos fornecedores de \textit{software} de se defenderem desses cibercriminosos. Isso porque, 
sabendo de quais vulnerabilidades os atacantes estão se aproveitando, as organizações saberiam o que precisa ser corrigido ou não na sua 
aplicação, e consequentemente poderiam lançar atualizações para corrigir as vulnerabilidades presentes, coibindo a ação desses 
cibercriminosos.
