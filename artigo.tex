\documentclass[brazil, utf8]{article}
\usepackage[utf8]{inputenc}
\usepackage{babel}

\begin{document}

\title{Prevenção de ataques \textit{zero-day} por meio de vulnerabilidades divulgadas}
\author{Alexandre Yuji Kajihara}
\date{2017}

\maketitle 

\section{Introdução}
O crime cibernético é uma forma de crime que inclui crimes que são cometidos através do uso da Internet ou de outras redes de
informáticas ~\cite{Fotiet:2015}. O aumento da disseminação de crimes por computador deve se a \textit{Return On Investement}
(ROI) derivado desse tipo de atividade criminosa ~\cite{Fotiet:2015}. Existem vulnerabilidades que podem valer mais de 100.000 dólares, 
dependendo da sua complexidade ~\cite{Bilge:2012}, um exemplo que podemos citar é as vulnerabilidades do tipo \textit{zero-day}, que 
iremos explicar o que elas são logo abaixo, em que devido a esse retorno de investimento os valores variam, na qual no segundo dia ela 
pode valer um quarto do que valia no \textit{zero-day} e no décimo dia o valor dela vale 1/1000 ~\cite{Fotiet:2015}. Além do que, essas 
vulnerabilidades são muito desejadas por agências governamentais, cibercriminosos e empresas de software ~\cite{Kumar:2016}.

Um dos tipos de ataque existente é o ataque de \textit{zero-day}, em que acredita-se que o período em que provavelemnte ocorreram as 
primeiras vulnerabilidades de \textit{zero-day} aconteceram nos anos 80 ~\cite{Fotiet:2015}. Esse ataque cibernético que explora uma 
vulnerabilidade que não foi divulgada publicamente, e esse ataques visam um número de organizações que possuem informações confidenciais 
que podem ser roubadas ~\cite{Bilge:2012}. Essas vulnerabilidades permanecem desconhecidas até por 2.5 anos ~\cite{Bilge:2012}. Enquanto 
a vulnerabilidade permanece desconhecida, o software obtido não pode ser corrigido e os produtos antivírus não conseguem detectar o 
ataque através da verificação baseada em assinatura ~\cite{Bilge:2012}. 

Uma maneira de prevenção é a implementação de sensores ou defesas cibernéticas, \textit{firewall} ~\cite{Last:2016}. Outras maneiras de 
prevenção seria por meio de \textit{Host Intrusion Protected Software} (HIPS), em que não se depende de assinaturas, já que há um 
monitoramente das atividades da máquina, usar um bom antivírus  protegendo de ataques conhecidos e desconhecidos e atualizar suas 
aplicações sempre que houver notificação de atualização de \textit{software} ~\cite{Kumar:2016}. Usuários que utilizam o antivírus podem 
ter mais cuidados com a segurança de seus computadores, e podem ser menos expostos aos ataques ~\cite{Bilge:2012}. Existem testes de 
penetração como o \textit{Web Application Penetration Testing} (WAPT)~\cite{Kumar:2016}. Apesar de todas essas maneiras de prevenir é 
díficil, porque a informação só estará disponível quando o ataque tiver sido concluído ~\cite{Kumar:2016}. Além disso, os fornecedores 
de \textit{software} dão preferência de corrigir vulnerabilidades que foram divulgadas ou estão prestes a declarar ~\cite{Bilge:2012}.

Como foi dito anteriormente, essas vulnerabilidades tem interessado pessoas, governos e empresas de software. No caso, dos ataques do 
tipo \textit{zero-day} existem três tipos de mercados que vendem vulnerabilidades de \textit{zero-day}, que no caso são o mercado negro, 
cinza e branco. O mercado branco é um mercado legal que não está escondido, em que as empresas de tecnologia pagam aos pesquisasdores 
dispostos a vender uma vulnerabilidade de \textit{zero-day} que descobriram. Empresas como o Google lançaram um programa para compra de 
vulnerabilidades de seus produtos ~\cite{Fotiet:2015}. 

O mercado negro, é um mercado de bens e serviços ilegais, onde as operações ocorrem através de contatos e vendas on-line, e fisicamente 
através de reuniões entre criminosos que compram. As vulnerabilidades ~\textit{zero-day} são vendidas no mercado negro em uam parte mais 
oculta. Um exemplo de um mercado negro é o \textit{Russian Business Network} sediada em São Petersburgo, na Rússia. 

O último mercado é o mercado cinza ou mercado governamental. O governo americano compra vulnerabilidades de \textit{zero-day} não para se 
proteger, mas sim para atacar. Na China, os alunos talentosos em ciência da computação e matemática são ensinados a espionagem industrial 
contra governos estrangeiros. Já na Índia, existe uma organização chamada \textit{National Technical Research Organization}, que uma lei 
autoriza, em caso de ataque, permite retaliar usando técnicas de \textit{hacking}. Além do que, o governo incentiva os jovens talentosos 
a entrar em um programa para proteger o país ~\cite{Fotiet:2015}.


\bibliographystyle{unsrt}
\bibliography{artigo}

\end{document}