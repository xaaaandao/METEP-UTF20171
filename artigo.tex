\documentclass[brazil, utf8]{article}
\usepackage[utf8]{inputenc}
\usepackage{babel}

\begin{document}

\title{Não sei}
\author{Alexandre Yuji Kajihara}
\date{2017}

\maketitle 

\section{Introdução}
O ataque de \textit{zero-day} é um ataque cibernético que explora uma vulnerabilidade que 
não foi divulgada publicamente, e com isso não há defesa contra esse ataque, consequentemente não
o \textit{software} não pode ser corrigido e os antivírus não detectam devido a não ter assinatura
~\cite{Bilge:2012}. Alguns tipos de ataques \texit{zero-day}, permanecem desconhecidos até por 2,5 
anos~\cite{Bilge:2012}. Quando as vulnerabilidades são descobertas ou são descritas em aviso 
público, os fornecedores de \textit{software} lançam atualizações ~\cite{Bilge:2012}. Além do que, 
um dos problemas é que depois de se lançados as atualizações, os usuários atrasam suas atualizações
~\cite{Bilge:2012}. O preço das bulnerabilidades como a de plataformas, (Windows, iOS, principais 
navegadores) pode execeder mais de 100.000 dólares, dependendo da complexidade ~\cite{Bilge:2012}.
Atualmente, as explorações são incorporadas em arquivos não executáveis, como *.pdf, *.doc, *.xlsx
~\cite{Bilge:2012}. Usuários que utilizam o antivírus podem ter mais cuidados com a segurança de 
seus computadores, e podem ser menos expostos aos ataques ~\cite{Bilge:2012}. Ataques \textit{
zero-day} são ataques chamados de ataques específicos já que que visam um número limitado de 
organizções que possuem informações confidenciais que podem ser roubadas ~\cite{Bilge:2012}. Além 
disso, os fornecedores de \texit{software} dão preferência de corrigir vulnerabilidades que foram 
divulgadas ou estão prestes a declarar ~\cite{Bilge:2012}.

Os defensores da redes estão sempre a dois passos de trás dos invasores, já que devem se defender 
de falhas que não foram descobertas ou foram descobertas por atacantes ~\cite{Last:2016}. A única 
vantagem é que os defensores sabem a localização provável e gravidade dessse ataque 
~\cite{Last:2016}. Uma maneira de se prevenir seria implementar sensores ou defesas cibernéticas, 
como \textit{firewall} ~\cite{Last:2016}.

Os ataques de \texit{zero-day} contra aplicações Web aumentaram ~\cite{Kumar:2016}. Além disso, 
existem testes como o de \textit{Web Application Penetration Testing} (WAPT), em que se as suas 
organizações não testam e não protegem de maneira adequadamente, terá seu aplicativo comprometido, 
ter dados roubados e danificar o desempenho da sua aplicação ~\cite{Kumar:2016}. È díficil de 
prevenir, porque a informação só estará disponível quando o  ataque tiver sido concluído 
~\cite{Kumar:2016}. Essas vulnerabilidades são muito desejadas por agências convernamentais, 
cibercriminosos e empresas de software ~\cite{Kumar:2016}. Essas falhas podem ser em
~\texit{software} ou sites ~\cite{Kumar:2016}. As maneira para evitar os ataques \textit{zero-day}
seria utilizando \texit{Host Intrusion Protected Software} (HIPS), em que não se depende de 
assinaturas, já que há um monitoramente das atividades da máquina. Usar um bom antivírus, 
protegendo de ataques conhecidos e desconhecidos. Atualizar suas aplicações, em que se houver 
notificação de atualização de \textit{software} atualize. Usar navegadores somente atualizados já 
que todos os navegadores empurram atualizações periódicas ~\cite{Kumar:2016}.

O crime cibernético é uma forma de crime que inclui crimes que são cometidos através do uso da
Internet ou de outras redes informáticas ~\cite{Armin:2015}. O aumento da dissemniação de crimes
por computador deve se a um \textit{Return On Investement} (ROI) derivado desse tipo de atividade 
criminosa ~\cite{Armin:2015}. As pessoas envolvidas em cibercrimes não entendem as ações ilegais   
realizadas com o auxílio de equipamentos de TI, e os mesmos se sentem que estão em um ambiente 
seguro com regras soltas ~\cite{Armin:2015}. As informaçẽos pessoais e corporativas representam um 
grande valor para os criminosos ~\cite{Armin:2015}. Acredita-se que o período em que provavelmente 
ocorreram as primeiras vulnerabilidades de \textit{zero-day}aconteceram nos anos 80 ~\cite{
Armin:2015}. Para atacar um sistema, é essencial estudar, compreender sua estrutura, identificar o 
tipo de proteção instalada e o sistema operacional, par aassim identifcar as vulnerabilidades
~\cite{Armin:2015}. Além disso, os valores dessas vulnerabilidades elas mudam no qual o no segundo 
ela vale um quarto do que do que \textit{zero-day}, e dez dias depois o valor dessa 
vulnerabilidade 1/1000 ~\cite{Armin:2015}. Existem fábricas de \texit{malware}, na qual empresas 
criam \texit{malwares} sob demanda em troca de uma taxa ~\cite{Armin:2015}. Existem três tipos de 
mercados. O mercado branco é um mercado legal que não está escondido, em que as empresas de 
tecnologia pagam aos pesquisasdores dispostos a vender uma vulnerabilidade de \texit{zero-day} que 
descobriram. Empresas como o Google lançaram um programa para compra de vulnerabilidades de seus 
produtos. Já o mercado negro, é um mercado de bens e serviços ilegais, onde as operações ocorrem  
através de contatos e vendas on-line, e fisicamente através de reuniões entre criminosos que 
compram. As vulnerabilidades ~\textit{zero-day} são vendidas no mercado negro em uam parte mais 
oculta. Um exemplo de um mercado negro é o \textit{Russian Business Network} sediada em São 
Petersburgo, na Rússia. Por fim, o último mercado é o mercado cinza ou mercado governamental. O 
governo americano comprar vulnerabilidades de \textit{zero-day} não para se proteger, mas sim para 
atacar. Na China, os alunos talentosos em ciência da computação e matemática são ensinados a 
espionagem industrial contra governos estrangeiros. Já na Índia, existe uma organização chamada 
\texit{National Technical Research Organization}, que uma lei autoriza, em caso de ataque, 
retaliar usando técnicas de \texit{hacking}. Além do que, o governo incentiva os jovens talentosos 
a entrar em um programa para proteger o país ~\cite{Armin:2015}.

A atividade de negociar explorações possui uma longa história, em que os criminosos utilizam cada vez mais disso
para próprio benefício econômico ~\cite{Egelman:2013}. Há prejuízos para a sociedade nos mercados de \textit{zero-day},
pois motivam os criminosos venda privada dessas falhas ~\cite{Egelman:2013}. Além disso, existem os \textit{Zero-Day 
Market} em que é a venda de detalhes de exploração ou \texit{bug} e o \texit{Vulnerability Rewards Program} esao 
ganhando muita popularidade, em que  é um programa pelo qual um fornecedor de software paga um pesquisador divulgar 
detalhes de um software relacionado à segurança, em que existem casos em que esses pesquisadores são contratados pelos 
fornecedores ~\cite{Egelman:2013}.Além disso, existem mercado de vulnerabilidades, em que os \textit{Hackers} são pagos 
pela sua fama e notoriedade ~\cite{Egelman:2013}. Esses mercados criam, incentivos atrativos para que pessoas mais 
inteligentes passem tempo procurando vulnerabilidades, já que quanto mais você olha, mais \textit{bugs} você encontrará 
~\cite{Egelman:2013}. Além do que, \textit{software} que são código aberto, ou de manutenção comunitária não existem 
fornecedores que estão dispostos a pagar pelas vulnerabilidades ~\cite{Egelman:2013}. Existem vendedores de 
vulnerabilidades que vendem para prejudicar um fabricante do software ~\cite{Egelman:2013}.

Ataques do tipo \textit{zero-day} são díficieis de analisar, porque em geral os dados não estão disponíveis de um 
ataque ser descoberto ~\cite{Bilge:2012}. O mercado de nova vulnerabilidade varia entre 5.000 e 250.000 dólares. 
Ataques de \textit{zero-day} foram o \textit{trojan} Hydraq e Aurora ~\cite{Bilge:2012}. Um ataque do tipo
\textit{zero-day} duram em média 312 dias e podem permanecer desconhecidos por até 2.5 anos ~\cite{Bilge:2012}.
Esse tipo de ataque começa com erros de programação ~\cite{Bilge:2012}. Além do que, existem fornecedores que aprendem 
uma vulnerabilidades antes dela serem exploradas e consideram baixa prioridade ~\cite{Bilge:2012}. Um problema que 
acontece é quando os fornecedores disponibilizam as atualizações, mas os usuário atrasam ela ~\cite{Bilge:2012}. Nos 
últimos anos, a maiorida das explorações são incorporadas em arquivos não executáveis como .pdf .doc. .xlsx 
~\cite{Bilge:2012}. Usuários que instalam o antivírus podem ter mais cuidados com a segurança de seus computadores, 
sendo menos expostos aos ataques ~\cite{Bilge:2012}. São ataques específicos, em que visam um número limitado de 
organizações que possuem informações confidenciais que podem ser roubadas ~\cite{Bilge:2012}. Enquanto o software tenha 
erros para novas vulnerabilidaes será um atividade lucrativa, e estaremos expostos a ataques de \textit{zero-day}
~\cite{Bilge:2012}.

%pesquisa sobre o WINE (World Intelligence Network Environment) OSVDB

\bibliographystyle{unsrt}
\bibliography{artigo}


\end{document}