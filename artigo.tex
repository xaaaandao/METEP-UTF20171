\documentclass[brazil, utf8]{article}
\usepackage[utf8]{inputenc}
\usepackage{babel}

\begin{document}

\title{Não sei}
\author{Alexandre Yuji Kajihara}
\date{2017}

\maketitle 

\section{Introdução}
Atualmente, os sistemas estão migrando para Web ~\cite{Cimapa2010} e são implementados com \textit{bugs}
 de software críticos que podem ser explorados de forma maliciosa ~\cite{Vieira2009}. Com isso as 
 aplicações Web estão sujeitos a ataque de hackers, tentando obter acesso não autorizado ou acessar 
 informações privadas ~\cite{Vieira2009}. Uma das técnicas de ataque mais populares é a de injeção SQL 
 em que se as consultas não forem validados, os atacantes podem ter acessos a dados não autorizados, 
 inserir e excluir dados, e até páginas de erro podem revelar qual banco de dados está sendo utilizado 
 e as tabelas presentes~\cite{Cimapa2010}. Para prevenir ataques desse tipo, existem técnicas que 
 obrigam o cliente a inserir dados corretos e que podem ser descobertos quando forem inseridos  valores 
 ilegais ~\cite{Vieira2009}. Além do que, desenvolvedores podem aplicar melhores práticas de codificação 
 ou executar testes de penetração para prevenir ataques em aplicações Web ~\cite{Vieira2009}. 
 Necessariamente, também existem alguns testes que podem ser feitos como o teste da caixa branca em que 
 se analisa códigos, porém se feito de forma exaustiva pode ser díficil e não encontrar todas as falhas 
 ~\cite{Vieira2009}. Já no teste da caixa preta são testes de penetração, em que o scanner pode não 
 reconhecer os componentes internos da aplicação Web. Porém, queremos prevenir possíveis e detectar 
 possíveis ataques ~\cite{Vieira2009}.


\bibliographystyle{unsrt}
\bibliography{artigo}


\end{document}